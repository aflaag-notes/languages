\documentclass[a4paper, 12pt]{report}

%%%%%%%%%%%%
% Packages %
%%%%%%%%%%%%

\usepackage{../Nyx/nyx-packages}
\usepackage{../Nyx/nyx-styles}
\usepackage{../Nyx/nyx-frames}
\usepackage{../Nyx/nyx-title}
\usepackage{../Nyx/nyx-macros}

%%%%%%%%%%%%%%
% Title-page %
%%%%%%%%%%%%%%

\logo{../Nyx/logo.png}

\institute{\curlyquotes{\hspace{0.25mm}Sapienza} Università di Roma}
\faculty{Ingegneria dell'Informazione,\\Informatica e Statistica}
\department{Dipartimento di Informatica}

\title{Linguaggi di Programmazione}
\subtitle{Appunti integrati con il libro "TODO", TODO 1, Autore 2, ...}

% \author{\textit{Author}\\TODO: DECOMMENTARE QUESTA SEZIONE}
% \author{\textit{Author}\\Simone Bianco}
\author{\textit{Author}\\Alessio Bandiera}
% \supervisor{Linus \textsc{Torvalds}}
% \context{Well, I was bored\ldots}

\date{\today}

%%%%%%%%%%%%
% Document %
%%%%%%%%%%%%

\begin{document}
    \maketitle

    % The following style changes are valid only inside this scope 
    {
        \hypersetup{allcolors=black}
        \fancypagestyle{plain}{%
        \fancyhead{}        % clear all header fields
        \fancyfoot{}        % clear all header fields
        \fancyfoot[C]{\thepage}
        \renewcommand{\headrulewidth}{0pt}
        \renewcommand{\footrulewidth}{0pt}}

        \romantableofcontents
    }

    \chapter*{Informazioni e Contatti}      % \chapter* makes this a "fake" chapter
    \markboth{Informazioni e Contatti}{}    % Manually sets \leftmark (current chapter name)
    \addcontentsline{toc}{chapter}{Informazioni e Contatti}     % Manually adds chapter to ToC
    
    \subsubsection{Prerequisiti consigliati:}
    \begin{itemize}
        \item Algebra
        \item TODO
    \end{itemize}

    \quad

    \subsubsection{Segnalazione errori ed eventuali migliorie:}
    
    Per segnalare eventuali errori e/o migliorie possibili, si prega di utilizzare il \textbf{sistema di Issues fornito da GitHub} all'interno della pagina della repository stessa contenente questi ed altri appunti (link fornito al di sotto), utilizzando uno dei template già forniti compilando direttamente i campi richiesti.

    Gli appunti sono in continuo aggiornamento, pertanto, previa segnalazione, si prega di controllare se l'errore sia ancora presente nella versione più recente.

    \quad

    \subsubsection{Licenza di distribuzione:}
    
    These documents are distributed under the \textbf{\href{https://www.gnu.org/licenses/fdl-1.3.txt}{GNU Free Documentation License}}, a form of copyleft intended to be used on manuals, textbooks or other types of document in order to assure everyone the effective freedom to copy and redistribute it, with or without modifications, either commercially or non-commercially.
    
    \quad

    \subsubsection{Contatti dell'autore e ulteriori link:}
    \begin{itemize}
        % \item TODO: DECOMMENTARE QUESTA SEZIONE

        % Simone
        % 
        % \item Altri appunti: \textbf{\href{https://github.com/Exyss/university-notes}{https://github.com/Exyss/university-notes}}
        % \item Github: \textbf{\href{https://github.com/Exyss}{https://github.com/Exyss}}
        % \item Email: \textbf{\href{mailto:bianco.simone@outlook.it}{bianco.simone@outlook.it}}
        % \item LinkedIn: \textbf{\href{https://www.linkedin.com/in/simone-bianco}{Simone Bianco}}

        % Alessio
        % 
        \item Github: \textbf{\href{https://github.com/ph04}{https://github.com/ph04}}
        \item Email: \textbf{\href{mailto:alessio.bandiera02@gmail.com}{alessio.bandiera02@gmail.com}}
        \item LinkedIn: \textbf{\href{https://www.linkedin.com/in/alessio-bandiera-a53767223/}{Alessio Bandiera}}
    \end{itemize}

    %%%%%%%%%%%%%%%%%%%%%

    \chapter{TODO}
    
    \section{TODO}

    \subsection{TODO}

    \begin{frameddefn}[label={peano}]{Assiomi di Peano}
        Gli \tbf{assiomi di Peano} sono 5 assiomi che definiscono l'insieme $\N$, e sono i seguenti:

        \begin{enumerate}[label=\roman*), font=\itshape]
            \item $0 \in \N$
            \item $\exists \func{\mathrm{succ}}{\N}{\N}$, o equivalentemente, $\forall x \in \N \quad \mathrm{succ}(x) \in \N$
            \item $\forall x, y \in \N \quad x \neq y \implies \mathrm{succ}(x) \neq \mathrm{succ}(y)$
            \item $\nexists x \in \N \mid \mathrm{succ}(x) = 0$
            \item $\forall S \subseteq \N \quad (0 \in S \land (\forall x \in S \quad \mathrm{succ}(x) \in S)) \implies S = \N$
        \end{enumerate}
    \end{frameddefn}

    \begin{example}[$\N$ di von Neumann]
        Una rappresentazione dell'insieme dei numeri naturali $\N$ alternativa alla canonica $$\N := \{0, 1, 2, \ldots \}$$ è stata fornita da John von Neumann. Indicando tale rappresentazione con $\aleph$, si ha che, per Neumann $$\begin{array}{c} 0_\aleph := \varnothing  = \{\} \\ 1_\aleph := \{ 0_\aleph \} = \{ \{ \}\} \\ 2_\aleph := \{0_\aleph, 1_\aleph\} = \{\{\}, \{\{ \}\}\} \\ \vdots \end{array}$$ e la funzione $\mathrm{succ}_\aleph$ è definita come segue $$\funcmap{\mathrm{succ}_\aleph}{\aleph}{\aleph}{x_\aleph}{\{\mu_\aleph \in \aleph \mid |\mu_\aleph| \le |x_\aleph|\}}$$ ed in particolare $\forall x_\aleph \in \aleph \quad |x_\aleph| + 1 = |\mathrm{succ}_\aleph(x_\aleph)|$

        È possibile verificare che tale rappresentazione di $\N$ soddisfa gli assiomi di Peano, in quanto

        \begin{enumerate}[label=\roman*), font=\itshape]
            \item $\varnothing := \varnothing \in \aleph$
            \item $\exists \func{\mathrm{succ}_\aleph}{\aleph}{\aleph}$, definita precedentemente
            \item $\forall x_\aleph, y_\aleph \in \aleph \quad x_\aleph \neq y_\aleph \implies |x_\aleph| \neq |y_\aleph| \implies  |\mathrm{succ}_\aleph(x_\aleph)| \neq |\mathrm{succ}_\aleph(y_\aleph)| \implies \mathrm{succ}_\aleph(x_\aleph) \neq \mathrm{succ}_\aleph(y_\aleph)$ 
            \item per assurdo, sia $x_\aleph \in \aleph$ tale che $\mathrm{succ}_\aleph(x_\aleph) = 0_\aleph := \varnothing$; per definizione $\mathrm{succ}_\aleph(x_\aleph) := \{\mu_\aleph \in \aleph \mid |\mu_\aleph| \le |x_\aleph|\}$, ma non esiste $\mu_\aleph \in \aleph$ con cardinalità minore o uguale 0, e dunque $\nexists x_\aleph \in \aleph \mid \mathrm{succ}_\aleph(x_\aleph) = 0_\aleph$
            \item per assurdo, sia $S \subseteq \aleph$ tale che $0_\aleph \in S$ e $\forall x_S \in S \quad \mathrm{succ}_\aleph(x_S) \in S$ ma $S \neq \aleph \iff \aleph - S \neq \varnothing$; allora $\exists x_\aleph \in \aleph - S$, e sicuramente $x_\aleph \neq 0_\aleph$ poiché $0_\aleph \in \aleph$, TODO DA FINIRE
        \end{enumerate}
    \end{example}

    \begin{framedprinc}[label={induction}]{Principio di Induzione}
        Sia $P$ una proprietà che vale per $n = 0$, e dunque $P(0)$ è vera; inoltre,  per ogni $n \in \N$ si ha che $P(n) \implies P(n + 1)$; allora, $P(n)$ è vera per ogni $n \in \N$.

        In simboli $$\forall P \quad (P(0) \land (\forall n \in \N \quad P(n) \implies P(n + 1))) \implies \forall n \in \N \quad P(n)$$
    \end{framedprinc}

    \begin{framedobs}{Quinto assioma di Peano}
        Si noti che il quinto degli assiomi di Peano (\cref{peano}) equivale al principio di induzione (\cref{induction}). Infatti, il quinto assioma afferma che qualsiasi sottoinsieme $S$ di $\N$ avente lo 0, e caratterizzato dalla chiusura sulla funzione di successore $\mathrm{succ}$, coincide con $\N$ stesso.
    \end{framedobs}

    \begin{frameddefn}{Algebra}
        Una \tbf{struttura algebrica}, o più semplicemente \tbf{algebra}, consiste di un insieme \tit{non vuoto}, talvolta chiamato \tbf{insieme sostegno} (\tit{carrier set} o \tit{domain}), fornito di una o più operazioni su tale insieme, quest'ultime caratterizzate da un numero finito di assiomi da soddisfare.

        Se $A$ è il carrier set, e $\gamma_1, \ldots \gamma_n$ sono delle operazioni definite su $A$, allora con $$(A, \gamma_1, \ldots, \gamma_n)$$ si indica l'algebra costituita da tali componenti.

        Con $\1$ verrà indicato un qualsiasi insieme tale che $\abs{\1} = 1$.
    \end{frameddefn}

    \begin{example}[Strutture algebriche con singola operazione]
        Esempi di strutture algebriche con un'operazione binaria sono i seguenti:

        \begin{itemize}
            \item semigruppi
            \item monoidi
            \item gruppi
            \item gruppi abeliani
        \end{itemize}
    \end{example}

    \begin{example}[Strutture algebriche con due operazioni]
        Esempi di strutture algebriche con due operazioni binarie sono i seguenti:

        \begin{itemize}
            \item semianelli
            \item anelli
            \item campi
        \end{itemize}
    \end{example}

    \begin{frameddefn}[label={inductive algebra}]{Algebra induttiva}
        Sia $A$ un insieme, e siano $\gamma_1, \ldots, \gamma_n$ funzioni definite su $A$ di arbitraria arietà; allora, $(A, \gamma_1, \ldots, \gamma_n)$ è definita \tbf{algebra induttiva} se si verificano le seguenti:

        \begin{enumerate}[label=\roman*), font=\itshape]
            \item $\gamma_1, \ldots, \gamma_n$ sono iniettive
            \item $\forall i, j \in [1, n] \mid i \neq j \quad \im(\gamma_i) \cap \im(\gamma_j) = \varnothing$, ovvero, le immagini delle funzioni sono a due a due disgiunte
            \item $\forall S \subseteq A \quad (\forall i \in [1, n], a_1, \ldots a_k \in S, k \in \N  \quad \gamma_i(a_1, \ldots, a_k) \in S) \implies S = A$, o equivalentemente, deve essere soddisfatto il \cref{induction}.
        \end{enumerate}
    \end{frameddefn}

    % \begin{framedobs}{Terzo assioma delle algebre induttive}
    %     Si noti che nel terzo assioma della \cref{inductive algebra} anche $S = \varnothing$ è un valido sottoinsieme di $A$, ma poiché non esistono $a_1, \ldots, a_k \in S \in \varnothing$, in esso ogni qualificazione è vera a vuoto.
    % \end{framedobs}

    \begin{example}[Numeri naturali]
        $(\N, +)$ non è un algebra induttiva, poiché esistono $x_1, x_2, x_3, x_4 \in \N$ con $x_1 \neq x_3$ e $x_2 \neq x_4$ tali che $x_1 + x_2 = x_3 + x_4$; ad esempio, 2 + 3 = 5 = 1 + 4, e 2 $\neq$ 1, 3 $\neq$ 4.
    \end{example}

    % \begin{example}[Algebra di Boole]
    %     Dato l'insieme $B = \{ \mathrm{true}, \mathrm{false}\}$, e la funzione $\lnot$ definita come segue $$\funcmap{\lnot}{B}{B}{x}{\left \{ \begin{array}{ll} \mathrm{false} & x = \mathrm{true} \\ \mathrm{true} & x = \mathrm{false} \end{array} \right.}$$ è possibile dimostrare che l'algebra $(B, \lnot)$ non è induttiva; infatti, nonostante $\lnot$ sia iniettiva, e la seconda proprietà della \cref{inductive algebra} sia vera a vuoto, si ha che $\forall x \in \varnothing \subseteq B \quad \nexists \lnot(x)$
    % \end{example}

    \begin{example}[Algebre induttive]
        Sia $\mathrm{zero}$ la funzione definita come segue $$\funcmap{\mathrm{zero}}{\1}{\N}{x}{0}$$ e si prenda in esame l'algebra $(\N, \mathrm{succ}, \mathrm{zero})$; allora si ha che

        \begin{enumerate}[label=\roman*), font=\itshape]
            \item $\mathrm{succ}$ e $\mathrm{zero}$ sono iniettive, poiché
                \begin{itemize}
                    \item $\mathrm{succ}$ è iniettiva per il terzo assioma di Peano (\cref{peano})
                    \item $\forall x, y \in \1 \quad \mathrm{zero}(x) = \mathrm{zero}(y) \implies x = y$ poiché $x, y \in \1 \implies x = y$ in quanto $\abs{\1} = 1$
                \end{itemize}
            \item $\im(\mathrm{succ}) \cap \im(\mathrm{zero}) = \left(\N - \{0\}\right) \cap \{0\} = \varnothing$
            \item TODO
        \end{enumerate}
    \end{example}

\end{document}
