\documentclass[a4paper, 12pt]{report}

%%%%%%%%%%%%
% Packages %
%%%%%%%%%%%%

\usepackage{../Nyx/nyx-packages}
\usepackage{../Nyx/nyx-styles}
\usepackage{../Nyx/nyx-frames}
\usepackage{../Nyx/nyx-title}
\usepackage{../Nyx/nyx-macros}

%%%%%%%%%%%%%%
% Title-page %
%%%%%%%%%%%%%%

\logo{../Nyx/logo.png}

\institute{\curlyquotes{\hspace{0.25mm}Sapienza} Università di Roma}
\faculty{Ingegneria dell'Informazione,\\Informatica e Statistica}
\department{Dipartimento di Informatica}

\title{Linguaggi di Programmazione}
\subtitle{Appunti integrati con il libro "TODO", TODO 1, Autore 2, ...}

% \author{\textit{Author}\\TODO: DECOMMENTARE QUESTA SEZIONE}
% \author{\textit{Author}\\Simone Bianco}
\author{\textit{Author}\\Alessio Bandiera}
% \supervisor{Linus \textsc{Torvalds}}
% \context{Well, I was bored\ldots}

\date{\today}

%%%%%%%%%%%%
% Document %
%%%%%%%%%%%%

\begin{document}
    \maketitle

    % The following style changes are valid only inside this scope 
    {
        \hypersetup{allcolors=black}
        \fancypagestyle{plain}{%
        \fancyhead{}        % clear all header fields
        \fancyfoot{}        % clear all header fields
        \fancyfoot[C]{\thepage}
        \renewcommand{\headrulewidth}{0pt}
        \renewcommand{\footrulewidth}{0pt}}

        \romantableofcontents
    }

    \chapter*{Informazioni e Contatti}      % \chapter* makes this a "fake" chapter
    \markboth{Informazioni e Contatti}{}    % Manually sets \leftmark (current chapter name)
    \addcontentsline{toc}{chapter}{Informazioni e Contatti}     % Manually adds chapter to ToC
    
    \subsubsection{Prerequisiti consigliati:}
    \begin{itemize}
        \item Algebra
        \item TODO
    \end{itemize}

    \quad

    \subsubsection{Segnalazione errori ed eventuali migliorie:}
    
    Per segnalare eventuali errori e/o migliorie possibili, si prega di utilizzare il \textbf{sistema di Issues fornito da GitHub} all'interno della pagina della repository stessa contenente questi ed altri appunti (link fornito al di sotto), utilizzando uno dei template già forniti compilando direttamente i campi richiesti.

    Gli appunti sono in continuo aggiornamento, pertanto, previa segnalazione, si prega di controllare se l'errore sia ancora presente nella versione più recente.

    \quad

    \subsubsection{Licenza di distribuzione:}
    
    These documents are distributed under the \textbf{\href{https://www.gnu.org/licenses/fdl-1.3.txt}{GNU Free Documentation License}}, a form of copyleft intended to be used on manuals, textbooks or other types of document in order to assure everyone the effective freedom to copy and redistribute it, with or without modifications, either commercially or non-commercially.
    
    \quad

    \subsubsection{Contatti dell'autore e ulteriori link:}
    \begin{itemize}
        % \item TODO: DECOMMENTARE QUESTA SEZIONE

        % Simone
        % 
        % \item Altri appunti: \textbf{\href{https://github.com/Exyss/university-notes}{https://github.com/Exyss/university-notes}}
        % \item Github: \textbf{\href{https://github.com/Exyss}{https://github.com/Exyss}}
        % \item Email: \textbf{\href{mailto:bianco.simone@outlook.it}{bianco.simone@outlook.it}}
        % \item LinkedIn: \textbf{\href{https://www.linkedin.com/in/simone-bianco}{Simone Bianco}}

        % Alessio
        % 
        \item Github: \textbf{\href{https://github.com/ph04}{https://github.com/ph04}}
        \item Email: \textbf{\href{mailto:alessio.bandiera02@gmail.com}{alessio.bandiera02@gmail.com}}
        \item LinkedIn: \textbf{\href{https://www.linkedin.com/in/alessio-bandiera-a53767223/}{Alessio Bandiera}}
    \end{itemize}

    %%%%%%%%%%%%%%%%%%%%%

    \chapter{TODO}
    
    \section{TODO}

    \subsection{TODO}

    \begin{frameddefn}[label={peano}]{Assiomi di Peano}
        Gli \tbf{assiomi di Peano} sono 5 assiomi che definiscono assiomaticamente l'insieme $\N$, e sono i seguenti:

        \begin{enumerate}[label=\roman*), font=\itshape]
            \item $0 \in \N$
            \item $\exists \func{\mathrm{succ}}{\N}{\N}$, o equivalentemente, $\forall x \in \N \quad \mathrm{succ}(x) \in \N$
            \item $\forall x, y \in \N \quad x \neq y \implies \mathrm{succ}(x) \neq \mathrm{succ}(y)$
            \item $\nexists x \in \N \mid \mathrm{succ}(x) = 0$
            \item $\forall S \subseteq \N \quad (0 \in S \land (\forall x \in S \quad \mathrm{succ}(x) \in S)) \implies S = \N$
        \end{enumerate}
    \end{frameddefn}

    \begin{framedprinc}[label={induction}]{Principio d'Induzione}
        Sia $P$ una proprietà, che vale per $n = 0$, dunque $P(0)$ è vera; inoltre,  per ogni $n \in \N$ si ha che $P(n) \implies P(n + 1)$. Allora, $P(n)$ è vera per ogni $n \in \N$. In simboli $$\forall P \quad (P(0) \land (\forall n \in \N \quad P(n) \implies P(n + 1))) \implies \forall n \in \N \quad P(n)$$
    \end{framedprinc}

    \begin{framedobs}{Quinto assioma di Peano}
        Si noti che il quinto assioma degli \cref{peano} equivale al \cref{induction}. Infatti, il quinto assioma afferma che qualsiasi sottoinsieme $S$ di $\N$ avente lo 0, e caratterizzato dalla chiusura sulla funzione di successore $\mathrm{succ}$, coincide con $\N$ stesso.
    \end{framedobs}

    \begin{frameddefn}{Struttura algebrica}
        Una \tbf{struttura algebrica}, o più semplicemente \tbf{algebra}, consiste di un insieme \tit{non vuoto}, talvolta chiamato \tbf{insieme sostegno} (\tit{carrier set} o \tit{domain}), fornito di una o più operazioni su tale insieme, ed un numero finito di assiomi che tali operazioni devono soddisfare.

        Se $A$ è il carrier set, e ad esempio $+$ è l'operazione binaria su $A$ definita come segue $$\func{+}{A \times A}{A}$$ allora con $(A, +)$ si indica l'algebra costituita da tali due elementi.
    \end{frameddefn}

    \begin{example}[Strutture algebriche]
        Esempi di strutture algebriche con un'operazione binaria sono i seguenti:

        \begin{itemize}
            \item semigruppi
            \item monoidi
            \item gruppi
            \item gruppi abeliani
        \end{itemize}

        mentre esempi di strutture algebriche con due operazioni binarie sono i seguenti:

        \begin{itemize}
            \item semianelli
            \item anelli
            \item campi
        \end{itemize}
    \end{example}

\end{document}
